\documentclass[11pt]{article}

\usepackage[margin = 1in]{geometry}
\usepackage{graphicx}              
\usepackage{amsmath}               
\usepackage{amsfonts}              
\usepackage{amsthm}                
\usepackage{amssymb}
\usepackage{mathrsfs}
\usepackage{url}
\usepackage{xcolor}
\usepackage{hyperref}
\usepackage{algpseudocode}
\usepackage[plain]{algorithm}
\usepackage{enumitem}
\usepackage{booktabs}
\usepackage{authblk}
\usepackage{tcolorbox}
\usepackage{mathtools}
\usepackage{bm}
\usepackage{dsfont}
\usepackage{tikz}
\usepackage{quantikz}
%\usepackage[T1]{fontenc}
%\usepackage{libertine}
%\usepackage[libertine]{newtxmath}


\hypersetup{
	unicode = true,
	colorlinks = true,
	citecolor = blue,
	filecolor = blue,
	linkcolor = blue,
	urlcolor = blue,
	pdfstartview = {FitH},
}

% theorem environments
\theoremstyle{plain}
\newtheorem{theorem}{Theorem}
\newtheorem{lemma}[theorem]{Lemma}
\newtheorem{corollary}[theorem]{Corollary}
\newtheorem{proposition}[theorem]{Proposition}
\theoremstyle{definition}
\newtheorem{definition}[theorem]{Definition}
\newtheorem{conjecture}[theorem]{Conjecture}
\newtheorem{example}[theorem]{Example}
\newtheorem*{remark}{Remark}
\newtheorem{note}{Note}
\newtheorem*{problem}{Problem}
\newtheorem*{fact}{Fact}



\algrenewcommand{\Require}{\item[\textbf{Input:}]}
\algrenewcommand{\Ensure}{\item[\textbf{Output:}]}

\newcommand{\wrt}{\vdash} 
\newcommand{\tildO}{\tilde{O}}

% roman numerals
\newcommand{\romnum}[1]{\romannumeral #1}
\newcommand{\Romnum}[1]{\uppercase\expandafter{\romannumeral #1}}

\newcommand{\todo}[1]{\textcolor{red}{TODO: #1}}

\DeclareMathOperator{\fieldchar}{char} % characteristic of a field
\DeclareMathOperator{\groupofend}{End} % endomorphism ring
\DeclareMathOperator{\tr}{Tr} % finite field trace
\DeclareMathOperator{\gal}{Gal} % Galois group
\DeclareMathOperator{\order}{ord} % order of an element
\DeclareMathOperator{\lcm}{lcm} % least common multiple
\DeclareMathOperator{\divisor}{div} % divisor on a curve
\DeclareMathOperator{\supp}{supp} % support of a divisor
\DeclareMathOperator{\norm}{N} % norm
\DeclareMathOperator{\negl}{negl} % norm
\DeclareMathOperator{\Res}{Res}
\DeclareMathOperator{\Aut}{Aut}
\DeclareMathOperator{\minpoly}{minpoly}
\DeclareMathOperator{\loglog}{loglog}
\DeclareMathOperator{\polylog}{polylog}
\DeclareMathOperator{\poly}{poly}
\DeclareMathOperator{\rev}{rev}
\DeclareMathOperator{\Hom}{Hom}
\DeclareMathOperator{\qft}{F}
\DeclareMathOperator{\E}{\mathbb{E}}


\DeclarePairedDelimiter{\abs}{\lvert}{\rvert}
\let\ket\relax
\DeclarePairedDelimiter{\ket}{\lvert}{\rangle}
\let\bra\relax
\DeclarePairedDelimiter{\bra}{\langle}{\rvert}
\DeclarePairedDelimiter{\lrang}{\langle}{\rangle}
\DeclarePairedDelimiter{\opnorm}{\lVert}{\rVert}


\def\Q{\mathbb{Q}}
\def\C{\mathbb{C}}
\def\K{\mathbb{K}}
\def\N{\mathbb{N}}
\def\R{\mathbb{R}}
\def\Z{\mathbb{Z}}
\def\F{\mathbb{F}}
\def\P{\mathbb{P}}
\def\MM{\mathsf{M}}
\def\CC{\mathsf{C}}
\def\lwe{\mathsf{LWE}}
\def\edcp{\mathsf{EDCP}}
\def\gen{\mathsf{Gen}}
\def\enc{\mathsf{Enc}}
\def\dec{\mathsf{Dec}}
\def\X{\mathcal{X}}
\def\SX{\mathcal{S(X)}}


\title{Quantum Learning With Errors}

\author{
	%% Javad Oilskin\thanks{Department of Computer Science, Ryerson University,
	%% (\tt{javad.doliskani@ryerson.ca}).}
}

\date{}
\setlength{\parindent}{0pt}
\sloppy




\begin{document}
\maketitle





%% ///////////////////////////////////////////////////////



\section{Preliminaries}
\label{sec:preli}



\subsection{Quantum Computation}

Our notations for quantum information mostly follow those of \cite{watrous2018theory}. The classical state of a register $\mathsf{X}$ is represented by a finite alphabet, say $\Sigma$. If the registers $\mathsf{X}_1, \dots, \mathsf{X}_n$ are represented by alphabets $\Sigma_1, \dots, \Sigma_n$ then the classical state of the tuple $(\mathsf{X}_1, \dots, \mathsf{X}_n)$ is represented by $\Sigma_1 \times \cdots \times \Sigma_n$. The complex Euclidean space associated with the register $\mathsf{X}$ is denoted by $\C^\Sigma$. For a complex Euclidean space $\X$, denote the unit sphere in $\X$ by $\SX$. A linear operator $\rho$ acting on $\X$ is called a density operator if $\rho$ is positive semidefinite with trace equal to $1$. The quantum state of the register $\mathsf{X}$ is represented by the set of density operators $\text{D}(\X)$.

We will use the Dirac notation for the elements of $\SX$. In particular, we denote the column vector $x \in \SX$ by $\ket{x}$ and the row vector $x^*$ by $\bra{x}$. A state $\rho$ is called pure if it can be written as $\rho = \ket{x}\bra{x}$, in which case we will simply write the state as $\ket{x}$. By the spectral theorem, every state $\rho$ is a linear combination of pure states. Therefore, a quantum state can also be represented as a linear combination
\[ \sum_{x} \alpha_x \ket{x}, \hspace*{1mm} \sum_{x} \abs{\alpha_x}^2 = 1. \]
We shall alternate between these equivalent representations of quantum state throughout this paper. The density operator representation is particularly useful when the underlying quantum state is not completely known. For example if, we only know that the system is in the state $\ket{\psi_x}$ with probability $p_x$ then the state of the system is described by the density operator
\[ \rho = \sum_{x} p_x \ket{\psi_x}\bra{\psi_x} = \E_x \Big[ \ket{\psi_x} \bra{\psi_x} \Big]. \]
More, generally, the density operator corresponding to a probability distribution $\gamma: \SX \rightarrow [0, 1]$ is defined as
\[ \rho_\gamma = \int_{\ket{\phi} \in \SX} \ket{\phi}\bra{\phi} d\gamma(\ket{\phi}) = \E_{\ket{\phi} \in \gamma} \Big[ \ket{\phi}\bra{\phi} \Big]. \]
For quantum public-key cryptography we will need a formal notion of quantum state discrimination. In particular, we need to formally define the notion of computational (in)distinguishability of quantum states. For our purposes, it is more convenient to define computational distinguishability for probability distributions over quantum states. The following is adapted from \cite[\S 3.3]{watrous2009zero}.
\begin{definition}
    Let $\X$ be a complex Euclidean space, and let $\gamma, \mu: \SX \rightarrow [0, 1]$ be probability distributions. Then $\gamma$ is said to be $(s, \epsilon)$-distinguishable from $\mu$ if there is a quantum measurement circuit $Q$ of size $s$ such that
    \[ \Big| \Pr_{\rho \in \gamma}[Q(\rho) = 1] - \Pr_{\rho \in \mu}[Q(\rho) = 1] \Big| \ge \epsilon. \]
\end{definition} 
Two distributions $\gamma, \mu$ are $(s, \epsilon)$-indistinguishable if they are not $(s, \epsilon)$-distinguishable.
\begin{definition}
    For each $n \in \N$, let $\X_n$ be a complex Euclidean space and let $\gamma_n, \mu_n: \mathcal{S}(\X_n) \rightarrow [0, 1]$ be probability distributions. Then the two ensembles $\{ \gamma_n \}_{n \in \N}$ and $\{ \mu_n \}_{n \in \N}$ are said to be polynomially quantum indistinguishable if for all polynomially bounded functions $s, p: \N \rightarrow \N$, the distributions $\gamma_n$ and $\mu_n$ are $(s(n), 1 / p(n))$-indistinguishable for almost all $n \in \N$.
\end{definition}
Two ensembles are called quantum computationally indistinguishable if they are polynomially quantum indistinguishable. The advantage of a polynomial-time quantum algorithm $Q$ in distinguishing between the distributions $\gamma_n$ and $\mu_n$ is defined as
\[ \delta_Q(\gamma_n, \mu_n) = \Big| \Pr_{\rho \in \gamma_n}[Q(\rho) = 1] - \Pr_{\rho \in \mu_n}[Q(\rho) = 1] \Big|. \]
Two ensembles $\{ \gamma_n \}$ and $\{ \mu_n \}$ are then quantum computationally indistinguishable if $\delta_Q(\gamma_n, \mu_n) = \negl(n)$ for all such $Q$ and almost all $n$. 



\subsection{Error reduction}

We can abstractly define the advantage of an algorithm $A$, regardless of $A$ being quantum or classical, in distinguishing between two probability distribution $P_1$ and $P_2$ as
\[ \delta_A(P_1, P_2) = \Big| \Pr_{x \in P_1}[A(x) = 1] - \Pr_{x \in P_2}[A(x) = 1] \Big|. \]
Two ensembles of distributions $\{ P_{1, n} \}$ and $\{ P_{2, n} \}$ are said to be poly-time indistinguishable if for any polynomial-time algorithm $A$ and any $\poly(n)$-bounded function $p$ we have $\delta_A(P_{1, n}, P_{2, n}) \le 1 / p(n)$ for large enough $n$. The following lemma follows from the triangle inequality.
\begin{lemma}[Hybrid lemma]
    Let $P_1, \dots, P_k$ be a sequence of probability distributions. Assume that $\delta_A(P_1, P_k) \ge \epsilon$ for some poly-time algorithm $A$. Then $\delta_A(P_i, P_{i + 1}) \ge \epsilon / k$ for some $1 \le i < k$.
\end{lemma}
Suppose an algorithm $A$ can distinguish between two distributions $P_1$ and $P_2$ with non-negligible advantage. A common technique to amplify the distinguishing advantage of $A$ is to sample enough times from the input distribution and then decide based on majority. A brief description of this technique, which we shall use several times in this paper, is as follows. First, we need the following well-known tail inequality.
\begin{lemma}[Hoeffding]
    Let $X_1, \dots, X_n$ be independent random variables with $X_i \in [a_i, b_i]$, and let $S = X_1, \cdots + X_n$. Then
    \begin{align*}
        \Pr[S - \E[S] \ge t] & \le e^{-2t^2 / \sum_i^n (b_i - a_i)^2}, \text{ and} \\
        \Pr[S - \E[S] \le -t] & \le e^{-2t^2 / \sum_i^n (b_i - a_i)^2}.
    \end{align*}
\end{lemma}
Now, assume $\delta_A(P_1, P_2) \ge 1 / p(n)$ for some polynomial $p(n)$, and let $P$ be the input distribution. Draw $m = 2np(n)^2$ samples from $P$, and let $X_i$ be a random variable representing the output of $A$ on input the $i$-th sample. Here, $X_i = 0$ (resp. $X_i = 1$) means $A$ has recognized the $i$-th sample to be from $P_1$ (resp. $P_2$). Let $S = X_1 + \cdots + X_m$. If $P = P_2$ then from the bound on $\delta_A$ we have $\E[S] \ge m / 2 + np(n)$. By Hoeffdings's inequality,
\begin{align*}
    \Pr\Big[ S \le \frac{1}{2} (m + np(n)) \Big]
    & = \Pr\Big[ S - \frac{m}{2} - np(n) \le -\frac{1}{2}np(n) \Big] \\
    & \le \Pr\Big[ S - \E[S] \le -\frac{1}{2}np(n) \Big] \\
    & \le e^{-n / 4}.
\end{align*}
Similarly, if $P = P_1$ then
\begin{align*}
    \Pr\Big[ S \ge \frac{1}{2} (m - np(n)) \Big]
    & \le \Pr\Big[ S - \E[S] \ge \frac{1}{2}np(n) \Big] \\
    & \le e^{-n / 4}.
\end{align*}
Therefore, by running $A$ on $m$ samples and counting the number of $1$'s we can tell, with probability exponentially close to $1$, whether $P = P_1$ or $P = P_2$.



\subsection{Learning With Errors}

In what follows, we briefly review the Learning With Error and the Extrapolated Dihedral Coset problems. Let $n \ge 1$, and $q = q(n) \ge 2$ be integers, and let $\chi$ be a probability distribution over $\Z$. For a random fixed $\bm{s} \in \Z_q^n$, denote by $A_{\bm{s}, \chi}$ the probability distribution over $\Z_q^n \times \Z_q$ defined as follows: choose $\bm{a} \in \Z_q^n$ uniformly at random, choose $e$ from according to $\chi$ and output  $(\bm{a}, \lrang{\bm{a}, \bm{s}} + e)$.
\begin{definition}[LWE, Search]
The search-$\lwe_{n, q, \chi}$ is the problem of recovering $\bm{s}$ given samples from the distribution $A_{\bm{s}, \chi}$. An algorithm $Q$ is said to solve $\lwe_{n, q, \chi}$ if $Q$  outputs $\bm{s}$ with probability at least $1 / \poly(n\log q)$ and has running time at most $\poly(n \log q)$.
\end{definition}
\begin{definition}[LWE, Decision]
    The decision-$\lwe_{n, q, \chi}$ problem is to distinguish between the distribution $A_{\bm{s}, \chi}$ and the uniform distribution over $\Z_q^n \times \Z_q$. An algorithm $Q$ is said to solve the desicion-$\lwe_{n, q, \chi}$ if it succeeds with advantage at least $1 / \poly(n\log q)$ and has running time at most $\poly(n\log q)$. 
\end{definition}
The distribution $\chi$ is called the error distribution and is usually chosen to be $\mathcal{D}_{\Z, \alpha q}$, the discrete Gaussian distribution centered around zero with standard deviation $\alpha q$. The parameter $\alpha \in (0, 1)$ is called the error rate. Let $n \ge 1$ and $q \ge 2$ be as above and let $r = r(n) < q$ be a positive integer. Let $\Sigma = \Z_r \times \Z_q^n$ and define the complex Euclidean space $\X = \C^\Sigma$. For a fixed $\bm{s} \in \Z_q^n$ define the probability distribution $\mu_{\bm{s}, r}: \SX \rightarrow [0, 1]$ as follows: choose $\bm{x} \in \Z_q^n$ uniformly at random and output the state
\[ \ket{\phi_{\bm{s}, r}(\bm{x})} = \frac{1}{\sqrt{r}} \sum_{j = 0}^{r - 1}\ket{j}\ket{\bm{x} + j\bm{s}}. \]
If we only have access to the output of $\mu_{\bm{s}, r}$, i.e., $\bm{x}$ is unknown, then the quantum system corresponding to the state $\ket{\phi_{\bm{s}, r}(\bm{x})}$ is described by the density operator
\[ \rho_{\bm{s}, r} = \frac{1}{q^n} \sum_{\bm{x} \in \Z_q^n} \ket{\phi_{\bm{s}, r}(\bm{x})} \bra{\phi_{\bm{s}, r}(\bm{x})} = \E_{\bm{x} \in U(\Z_q^n)} \Big[ \ket{\phi_{\bm{s}, r}(\bm{x})} \bra{\phi_{\bm{s}, r}(\bm{x})} \Big], \]
where $U(\Z_q^n)$ is the uniform distribution over $\Z_q^n$. Therefore, the output of the distribution $\mu_{\bm{s}, r}$ is always described by the density operator $\rho_{\bm{s}, r}$. In other words, the distribution $\mu_{\bm{s}, r}$ provides \textit{copies} of the state $\rho_{\bm{s}, r}$.
\begin{definition}[EDCP, Search]
    Let $n$, $q$ and $r$ be defined as above. The search-$\edcp_{n, q, r}$ is the problem of recovering $\bm{s}$ given samples from the distribution $\mu_{\bm{s}, r}$. A quantum algorithm $Q$ is said to solve $\edcp_{n, q, r}$ if it outputs $\bm{s}$ with probability at least $1 / \poly(n\log q)$ and has running time at most $\poly(n\log q)$.
\end{definition}
\begin{definition}[EDCP, Decision]
    \label{def:d-edcp}
    Let $n$, $q$ and $r$ be defined as above. Define the probability distribution $\gamma_r: \SX \rightarrow [0, 1]$ by choosing $(j, \bm{x}) \in \Z_r \times \Z_q^n$ uniformly at random and outputting the state $\ket{j}\ket{\bm{x}}$. The decision-$\edcp_{n, q, r}$ is the problem of distinguishing between the distributions $\mu_{\bm{s}, r}$ and $\gamma_r$.
\end{definition}
A quantum algorithm $Q$ is said to solve the decision-$\edcp_{n, q, r}$ if it succeeds with advantage at least $1 / \poly(n\log q)$ and has running time at most $\poly(n\log q)$. The density operator corresponding to the output of the distribution $\gamma_r$ in Definition \ref{def:d-edcp} is
\[ \rho = \frac{1}{rq^n} \sum_{j = 0}^{r - 1} \sum_{\bm{x} \in \Z_q^n}  \ket{j}\ket{\bm{x}} \bra{j}\bra{\bm{x}} = \E_{(j, \bm{x}) \in U(\Z_r \times \Z_q^n)} \Big[ \ket{j}\ket{\bm{x}} \bra{j}\bra{\bm{x}} \Big] = \mathds{1}_{\X}, \]
where $U(\Z_r \times \Z_q^n)$ is the uniform distribution over $\Z_r \times \Z_q^n$. Therefore, decision-$\edcp_{n, q, r}$ is the problem of distinguishing between the same number of copies of the states $\rho_{\bm{s}, r}$ and $\mathds{1}_{\X}$. 


%% ///////////////////////////////////////////////////////



\section{A Search to Decision Reduction}

In this section, we give a search-to-decision reduction for $\edcp$. The reduction works for a large class of moduli $q$. The technique we use is inspired by the one used in \cite{micciancio2012trapdoors} for a search-to-decision reduction for LWE.
\begin{lemma}
    \label{lem:small-r}
    For any $r' \le r$, given access to the distribution $B_{\bm{s}, r}$, we can efficiently sample from the distribution $B_{\bm{s}, r'}$. 
\end{lemma}
\begin{proof}
    If $r' > r / 2$ then a simple indicator function can be used to to produce sample from $B_{\bm{s}, r'}$. More precisely, define the function $f: [0, r) \rightarrow \{ 0, 1 \}$ by
    \[ f(x) = 
    \begin{cases}
        1 & \text{if } x < r' \\
        0 & \text{otherwise}.
    \end{cases} \]
    Then applying the transformation $\ket{j}\ket{\bm{a}}\ket{0} \mapsto \ket{j}\ket{\bm{a}}\ket{f(j)}$, where $\bm{a} \in \Z_q^n$, to $\rho_{\bm{s}, r}$ and measuring the last register results in the state $\rho_{\bm{s}, r'}$ with probability at least $1 / 2$. If the measurement outcome is not $1$ then we repeat the above process.

    If $r' \le r / 2$ then we proceed as follows. Consider the measurement $\mu$ on the space $\X$ defined by the operators
    \[ M_a = \sum_{b = 0}^{\ell - 1} \ket{b}\bra{ar' + b} \otimes \mathds{1}, \]
    where $\ell = r'$ for $0 \le a < \lfloor r / r'  \rfloor$ and $\ell = r - r'$ for $a = \lfloor r / r'  \rfloor$. This measurement can be implemented efficiently \cite{kaye2007introduction}. If we perform $\mu$ on a sample $\rho_{\bm{s}, r}$ from $B_{\bm{s}, r}$ the probability of observing the outcome $a$ is
    \begin{align*}
        \tr(M_a^*M_a \rho_{\bm{s}, r})
        & = \frac{1}{q^n} \sum_{\bm{x} \in \Z_q^n} \tr(M_a \ket{\phi_{\bm{s}, r}(\bm{x})} \bra{\phi_{\bm{s}, r}(\bm{x})} M_a^*) \\
        & = \frac{1}{q^nr} \sum_{\bm{x} \in \Z_q^n} \sum_{b, c = 0}^{\ell - 1} \tr(\ket{b}\ket{\bm{x} + (ar' + b)\bm{s}} \bra{c}\bra{\bm{x} + (ar' + c)\bm{s}}) \\
        & = \frac{1}{q^nr} \sum_{\bm{x} \in \Z_q^n} \sum_{b, c = 0}^{\ell - 1} \tr(\ket{b}\bra{c} \otimes \ket{\bm{x} + (ar' + b)\bm{s}} \bra{\bm{x} + (ar' + c)\bm{s}}) \\
        & = \frac{\ell}{r},
    \end{align*}
    and the post-measurement state corresponding to this outcome is
    \begin{align*}
        \frac{M_a \rho_{\bm{s}, r} M_a^*}{(\ell / r)}
        & = \frac{1}{q^nr} \sum_{\bm{x} \in \Z_q^n} \sum_{b, c = 0}^{\ell - 1} \ket{b}\ket{\bm{x} + (ar' + b)\bm{s}} \bra{c}\bra{\bm{x} + (ar' + c)\bm{s}} \\
        & = \frac{1}{q^nr} \sum_{\bm{x} \in \Z_q^n} \sum_{b, c = 0}^{\ell - 1} \ket{b}\ket{\bm{x} + b\bm{s}} \bra{c}\bra{\bm{x} + c\bm{s}} \\
        & = \rho_{\bm{s}, \ell},
    \end{align*}
    So if the outcome is $a \in [0, \lfloor r / r'  \rfloor)$ we obtain the state $\rho_{\bm{s}, r'}$, which is what we are looking for. Therefore, the probability of obtaining the desired state after one measurement is
    \[ \lfloor r / r' \rfloor \frac{\ell}{r} = \lfloor r / r' \rfloor \frac{r'}{r} \ge \left( \frac{r}{r'} - 1 \right)\frac{r}{r'} = 1 - \frac{r'}{r} \ge \frac{1}{2}. \]
    If the measurement outcome is $a = \lfloor r / r' \rfloor$ then we repeat the above process.
\end{proof}
\begin{theorem}
    Let $q = p_1^{e_1} \cdots p_\ell^{e_\ell}$ be the prime factorization of $q$ and assume that the primes $p_i$ are of size $\poly(n)$. Then there is a polynomial-time quantum reduction from solving search-$\edcp_{n, q, r}$ to solving decision-$\edcp_{n, q, r'}$ for any $r' \le r$ such that $r' \le p_i^{e_i}$ for all $i$. 
\end{theorem}
\begin{proof}
    Let $D$ be an oracle for solving decision-$\edcp_{n, q, r'}$. Given samples from the distribution $B_{\bm{s}, r}$, we will use $D$ to recover $\bm{s} \bmod p^{e_i}$ for each $i$, and then assemble the results using the Chinese remainder theorem to recover $\bm{s} \bmod q$. We shall compute $\bm{s} \bmod p_1^{e_1}$, the algorithm is the same for the other $p_i$. Let $p = p_1$ and $e = e_1$.

    The first step is to construct the distribution $B_{\bm{s}, r'}$ from the distribution $B_{\bm{s}, r}$. This can be efficiently done using Lemma \ref{lem:small-r}. Now, from $B_{\bm{s}, r'}$ we construct the distribution $B_{\bm{s}, r'}^k$ for all $k = 1, \dots, e$. Given a sample $\rho_{\bm{s}, r'}$ from $B_{\bm{s}, r'}$, a sample from $B_{\bm{s}, r'}^k$ is constructed by computing $j \bmod p^k$ into an auxiliary register and then discarding the register. More precisely, if we denote by  $\ket{\phi_{\bm{s}, r'}^k(\bm{x})}$ the result of 
    \begin{align}
        \ket{\phi_{\bm{s}, r'}(\bm{x})}\ket{0}
        & \mapsto \frac{1}{\sqrt{r'}} \sum_{j = 0}^{r' - 1} \ket{j}\ket{\bm{x} + j\bm{s}}\ket{j \bmod p^k} \label{equ:r-1}  \\
        & \mapsto \frac{1}{\sqrt{r_k}} \sum_{j = 0}^{r_k - 1} \ket{jp^k + c}\ket{\bm{x} + (jp^k + c)\bm{s}}, \nonumber
    \end{align}
    where $0 \le r_k \le \lfloor r' / p^k \rfloor$ and the random constant $0 \le c \le p^k - 1$ depend on the outcome of measuring the last register, then
    \[ \rho_{\bm{s}, r'}^k = \frac{1}{q^n} \sum_{\bm{x} \in \Z_q^n} \ket{\phi_{\bm{s}, r'}^k(\bm{x})} \bra{\phi_{\bm{s}, r'}^k(\bm{x})} \]
    is a sample from $B_{\bm{s}, r'}^k$. For $k = 0$ we have $j = 0 \bmod p^0$ for all $j$, so $\rho_{\bm{s}, r'}^0 = \rho_{\bm{s}, r'}$ and the distribution $B_{\bm{s}, r'}^0$ is the same as $B_{\bm{s}, r'}$. For $k = e$, since $r' \le p^e$, discarding the last register in \eqref{equ:r-1} collapses the state $\rho_{\bm{s}, r'}$ to $\mathds{1}$. Therefore, by a hybrid argument there is a $1 \le k \le e$ such that $\edcp_{n, q, r'}$ can distinguish between $B_{\bm{s}, r'}^{k - 1}$ and $B_{\bm{s}, r'}^k$ with non-negligible advantage. Using the amplification technique of Section ?? we can assume that the distinguishing advantage is exponentially close to $1$. Let $\bm{s} = (s_1, \dots, s_n)$. Given samples from $B_{\bm{s}, r'}^{k - 1}$ we can compute $s_1 \bmod p$ as follows. Consider the state $\ket{\phi_{\bm{s}, r'}^{k - 1}(\bm{x})}$ where $\bm{x} = (x_1, \dots, x_n)$, and let $y \in \Z_p$. If we perform the transformation $\ket{j}\ket{\bm{a}}\ket{0} \mapsto \ket{j}\ket{\bm{a}}\ket{a_1 - jy \bmod p^k}$ on $\ket{\phi_{\bm{s}, r'}^{k - 1}(\bm{x})}$ we have
    \[ \ket{\phi_{\bm{s}, r'}^{k - 1}(\bm{x})}\ket{0} \mapsto \frac{1}{\sqrt{r_{k - 1}}} \sum_{j = 0}^{r_{k - 1} - 1} \ket{jp^{k - 1} + c}\ket{\bm{x} + (jp^{k - 1} + c)\bm{s}}\ket{x_1 + (jp^{k - 1} + c)(s_1 - y) \bmod p^k}. \]
    If we measure the last register, the resulting state will be a sample from $B_{\bm{s}, r'}^{k - 1}$ or $B_{\bm{s}, r'}^k$ depending on whether $s_1 = y$ or $s_1 \ne y \bmod p$:
    \begin{itemize}
    \item $y = s_1 \bmod p$. In this case, the value of the last register is $x_1 \bmod p$ which is not entangled with the first two registers. So we obtain the original sample from $B_{\bm{s}, r'}^{k - 1}$.
    \item $y \ne s_1 \bmod p$. In this case, the new state contains the terms with $j = (c_1 - x_1) / (s_1 - y) - c \bmod p$ for a random constant $0 \le c_1 \le p^k - 1$. So the new state is
    \begin{align*}
        \ket{\psi}
        & = \frac{1}{\sqrt{r_k}} \sum_{j = 0}^{r_k - 1} \ket{(jp + c_2)p^{k - 1} + c}\ket{\bm{x} + ((jp + c_2)p^{k - 1} + c)\bm{s}} \\
        & = \frac{1}{\sqrt{r_k}} \sum_{j = 0}^{r_k - 1} \ket{jp^k + c_2p^{k - 1} + c}\ket{\bm{x} + (jp^k + c_2p^{k - 1} + c)\bm{s}}
    \end{align*}
    where $0 \le c_2 \le p - 1$ is a random constant and $0 \le r_k \le \lfloor r_{k - 1} / p \rfloor$. The mixed state corresponding to $\ket{\psi}$ is clearly a sample from $B_{\bm{s}, r'}^k$.
    \end{itemize}
    Therefore, using $D$ we can find out whether $y = s_1 \bmod p$. By trying every $y \in \Z_p$ we can recover $s_1 \bmod p$. Assume we have recovered the first $h$ digits of $s_1$ in base $p$, that is we have computed $0 \le \tilde{s}_1 < p^h$ such that $\tilde{s}_1 = s_1 \bmod p^h$. To compute the $(h + 1)$-th digit, we perform the transformation
    \[ \ket{j}\ket{\bm{a}}\ket{0} \mapsto \ket{j}\ket{\bm{a}}\ket{(a_1 - j\tilde{s}_1) / p^h - jy \bmod p}  \]
    on the samples from $B_{\bm{s}, r'}^{k - 1}$ and repeat the above procedure.
\end{proof}
\begin{corollary}
    Let $q = p_1^{e_1} \cdots p_\ell^{e_\ell}$ be the prime factorization of $q$ and assume that the primes $p_i$ are of size $\poly(n)$. If $r \le p_i^{e_i}$ for all $1 \le i \le \ell$ then there is a polynomial-time quantum reduction from solving search-$\edcp_{n, q, r}$ to solving decision-$\edcp_{n, q, r}$. In particular, if $q$ is a prime power then search-$\edcp_{n, q, r}$ and decision-$\edcp_{n, q, r}$ are quantum polynomial-time equivalent.
\end{corollary}
It follows from the above corollary that for the two special cases $q = 2^e$ and $q = p$, where $p$ is $\poly(n)$-bounded prime, the search-$\edcp$ and decision-$\edcp$ are polynomial-time equivalent without loss of parameters. 



%% ///////////////////////////////////////////////////////



\section{A Better Decision Problem}

In this section, we propose an $\edcp$ decision problem that will be more suitable for applications than the one defined in Section \ref{sec:preli}. We assume that the modulus $q$ has $\poly(n)$-bounded prime factors. Perhaps the new decision problem is best understood for a prime modulus $q$. So let us assume, for now, that $q$ is a $\poly(n)$-bounded prime. 

Define the distribution $\tilde{B}_{\bm{s}, r}$ on the unit sphere $\SX$ by choosing $\bm{x} \in \Z_q^n$ and $t \in \Z_q {\setminus} \{ 0 \}$ uniformly at random and outputting the state
\begin{equation}
    \label{equ:new-dec}
    \ket{\tilde{\phi}_{\bm{s}, r}(\bm{x})} = \frac{1}{\sqrt{r}} \sum_{j = 0}^{r - 1} \omega_q^{jt} \ket{j}\ket{\bm{x} + j\bm{s}}.
\end{equation}
The new decision problem is to distinguish between the distributions $B_{\bm{s}, r}$ and $\tilde{B}_{\bm{s}, r}$. The motivation behind this new definition is that the state \eqref{equ:new-dec} can be efficiently transformed to a \textit{shifted} LWE sample $(\bm{a}, \lrang{\bm{a}, \bm{s}} + e + t)$ where $e$ is sampled from $\mathcal{D}_{\Z, q / r}$. For a large enough $q$, this pair is closer to a uniformly random element of $\Z_q^n \times \Z_q$ than an LWE sample. An instance of the above decision problem then translates to an instance of the LWE decision problem. The transformation of \eqref{equ:new-dec} into a shifted LWE sample can be done using the technique in \cite{brakerski2018learning}: given the state $\ket{\tilde{\phi}_{\bm{s}, r}(\bm{x})}$, we first apply the transformation $\ket{j} \mapsto \ket{j - \lfloor (r - 1) / 2 \rfloor}$ to the first register to obtain the state
\begin{equation}
    \label{equ:symm-edcp}
    \frac{1}{\sqrt{r}} \sum_{j = -\lfloor (r - 1) / 2 \rfloor}^{\lceil (r - 1) / 2 \rceil} \omega_q^{jt} \ket{j}\ket{\bm{x} + j\bm{s}},
\end{equation}
where we have again denoted the random element $\bm{x} + \lfloor (r - 1) / 2 \rfloor \bm{s} \in \Z_q^n$ by $\bm{x}$. Next, using the quantum rejection sampling we can, with probability $\Omega(\sigma / r) = \Omega(1 / \sqrt{\kappa})$, transform \eqref{equ:symm-edcp} into 
\begin{equation}
    \label{equ:symm-eg}
    \sum_{j = -\lfloor (r - 1) / 2 \rfloor}^{\lceil (r - 1) / 2 \rceil} \omega_q^{jt} g_\sigma(j) \ket{j}\ket{\bm{x} + j\bm{s}},
\end{equation}
where $g_\sigma(x) = \exp(-\pi x^2 / \sigma^2)$ is the Gaussian distribution. Now if we apply the transform $\qft_q \otimes \qft_{q^n}$ to \eqref{equ:symm-eg} and measure the last register we obtain the state
\[ \ket{\psi} =  \frac{1}{\sqrt{q}} \sum_{y \in \Z_q} \sum_{j = -\lfloor (r - 1) / 2 \rfloor}^{\lceil (r - 1) / 2 \rceil} \omega_q^{j(\lrang{\bm{a}, \bm{s}} + y + t)} g_\sigma(j) \ket{y},\]
where $\bm{a} \in \Z_q^n$ is uniformly random and known. We have
\begin{align*}
    \ket{\psi}
    & \approx_\epsilon \sum_{y \in \Z_q} \sum_{j \in \Z} \omega_q^{j(\lrang{\bm{a}, \bm{s}} + y + t)} g_\sigma(j) \ket{y} \tag{by Lemma ??} \\
    & = \sum_{y \in \Z_q} \sum_{j \in \Z} g_{1/\sigma} \Big( j + \frac{\lrang{\bm{a}, \bm{s}} + y + t}{q} \Big) \ket{y} \tag{by Lemma ??} \\
    & = \sum_{e \in \Z} g_{1/\sigma} \Big( \frac{e}{q} \Big) \ket{\lrang{-\bm{a}, \bm{s}} + e - t \bmod q} \tag{$e = jq + \lrang{\bm{a}, \bm{s}} + t + y$} \\
    & \approx_\epsilon \sum_{e \in \Z_q} g_{1/\sigma} \Big( \frac{e}{q} \Big) \ket{\lrang{-\bm{a}, \bm{s}} + e - t} \tag{by Lemma ??}.
\end{align*}
Measuring the above state, we obtain a pair $(-\bm{a}, \lrang{-\bm{a}, \bm{s}} + e - t)$ where $e$ is sampled from $\mathcal{D}_{\Z, q / \sigma}$. For a general modulus $q$, an immediate generalization of the decision problem would be to just replace the prime modulus with a general one, and the above transformation to an LWE sample goes through without any change. However, for such a generalization, it is not clear how to reduce the search problem to the decision problem when $q$ is super-polynomially large in $n$.
\begin{definition}[EDCP, Decision]
    Let $p \mid q$ be a prime. Define the distribution $B_{\bm{s}, r, p}$ on the unit sphere $\SX$ by choosing $\bm{x} \in \Z_q^n$ and $t \in \Z_p {\setminus} \{ 0 \}$ uniformly at random and outputting the state
    \begin{equation}
        \ket{
            \phi_{\bm{s}, r, p}(\bm{x})} = \frac{1}{\sqrt{r}} \sum_{j = 0}^{r - 1} \omega_p^{jt} \ket{j}\ket{\bm{x} + j\bm{s}}.
    \end{equation}
    The decision-$\edcp_{n, q, r}$ is the problem of distinguishing between the distribution $B_{\bm{s}, r}$ and any distribution in the set $\{ B_{\bm{s}, r, p} \}_{p \mid q}$.
\end{definition}
\begin{theorem}
    Assume that all the prime factors of $q$ are $\poly(n)$-bounded. Then there is a polynomial-time quantum reduction from solving search-$\edcp_{n, q, r}$ to solving decision-$\edcp_{n, q, r}$.
\end{theorem}
\begin{proof}
    Let $q = p_1^{e_1} \cdots p_\ell^{e_\ell}$ be the prime factorization of $q$. Let $D$ be an oracle for solving decision-$\edcp_{n, q, r}$. The idea is to use $D$ to find $\bm{s} \bmod p_i^{e_i}$ for all $i$ and then reconstruct $\bm{s} \bmod q$ using the Chinese remainder theorem. Let $\bm{s} = (s_1, \dots, s_n)$. We show how to recover $s_1 \bmod p_1^{e_1}$, the other values $s_i \bmod p_j^{e_j}$ can be computed similarly. Set $p = p_1$ and $e = e_1$.
    For any $y \in \Z_p$ and nonzero $c \in \Z_p$ define the following unitary on $\X$
    \[ U_{c, y} \ket{j}\ket{\bm{a}} = \omega_p^{(a_1 - jy)c}\ket{j}\ket{\bm{a}}, \]
    where $a_1$ is the first coordinate of $\bm{a}$. Given a sample $\rho_{\bm{s}, r}$ from $B_{\bm{s}, r}$, fix $y \in \Z_p$ and select a fresh nonzero $c \in \Z_p$ uniformly at random. Then we have
    \[ U_{c, y}\ket{\phi_{\bm{s}, r}(\bm{x})} = \frac{1}{\sqrt{r}} \omega_p^{x_1}\sum_{j = 0}^{r - 1} \omega_p^{j(s_1 - y)c} \ket{j}\ket{\bm{x} + j\bm{s}}. \]
    Therefore, ignoring the global phase, if $s_1 \ne y \bmod p$ then $U_{c, y} \rho_{\bm{s}, r} U_{c, y}^*$ is a sample from $B_{\bm{s}, r, p}$, otherwise $U_{c, y} \rho_{\bm{s}, r} U_{c, y}^* = \rho_{\bm{s}, r}$. So, the oracle $D$ could tell us which is the case. Trying all $y \in \Z_p$ we can find $s_1 \bmod p$. Now assume we have recovered $\tilde{s}_1 = s_1 \bmod p^k$ for $k < e$. To compute $s_1 \bmod p^{k + 1}$, we can modify the unitary $U_{c, y}$ as 
    \[ U_{c, y, k} \ket{j}\ket{\bm{a}} = \omega_{p^{k + 1}}^{(a_1 - j\tilde{s}_1 - jp^ky)c}\ket{j}\ket{\bm{a}}. \]
    To see how $U_{c, y, k}$ acts on a sample $\rho_{\bm{s}, r}$ from $B_{\bm{s}, r}$, let $s_{1, k + 1}$ be the $(k + 1)$-th digit of $s_1$ in base $p$. Then
    \begin{align*}
        U_{c, y, k} \ket{j}\ket{\bm{x} + j\bm{s}}
        & = \omega_{p^{k + 1}}^{(x_1 + js_1 - j\tilde{s}_1 - jp^ky)c}\ket{j}\ket{\bm{x} + j\bm{s}} \\
        & = \omega_{p^{k + 1}}^{x_1} \omega_p^{j(s_{1, k + 1} - y)c}\ket{j}\ket{\bm{x} + j\bm{s}}.
    \end{align*}
    Therefore, repeating the above procedure, we can recover $s_{1, k + 1}$. This completes the proof.
\end{proof}



%% ///////////////////////////////////////////////////////



\section{Quantum Public-Key Cryptosystem}

A quantum public-key cryptosystem is, similar to a classical system, a set of three algorithms:
\begin{itemize}[itemsep = 1pt]
\item $\gen(1^n)$ generates a public-key $pk$ and a secret-key $sk$, based on the security parameter $n$.
\item $\enc(pk, m)$, outputs a ciphertext $c$ for a given public-key $pk$ and message $m$.
\item $\dec(sk, c)$, outputs a message $m$ for a given secret-key $sk$ and ciphertext $c$.
\end{itemize}
The output pair $(pk, sk)$ of the $\gen$ algorithm for a quantum system is a consists usually of a quantum state and a classical state, respectively. In particular, the public-key $pk$ is a quantum state that is generated using a classical key $sk$. The algorithm $\enc$ encrypts the message $m$, which is classical information, using the quantum state $pk$. The output $c$ of $\enc$ is a quantum state. The algorithm $\dec$ uses the key $sk$ to decrypt the quantum state into a classical message $m$.

For the security parameter $n$, we set the parameters for public key system as follows. We choose $q = \poly(n)$ such that $q = 2^s$ for some integer $s > 0$. We also set $r = 2^{s'}$ where $s' < s$. The main reason for these choices of parameters is efficiency, as we shall explain at the end of this section. In what follows we describe our cryptosystem for encrypting a one-bit message $b \in \{ 0, 1 \}$.

\vspace*{\topskip}

$\gen(1^n)$: Select $\bm{x}, \bm{s} \in \Z_q^n$ uniformly at random. Apply the transformation $\qft_r \otimes \mathds{1}$ to the register $\ket{0} \ket{\bm{x}}$ to obtain the state $\ket{\psi_1} = \frac{1}{\sqrt{r}} \sum_{j = 0}^{r - 1} \ket{j} \ket{\bm{x}}$. Apply the transformation $\ket{j}\ket{\bm{x}} \mapsto \ket{j}\ket{\bm{x} + j\bm{s}}$ to $\ket{\psi_1}$ to obtain the state
\[ \ket{\phi_{\bm{s}, r, 0}(\bm{x})} = \frac{1}{\sqrt{r}} \sum_{j = 0}^{r - 1} \ket{j} \ket{\bm{x} + j\bm{s}}. \]
Return the public-key, secret-key pair $(pk, sk) = (\ket{\phi_{\bm{s}, r, 0}(\bm{x})}, \bm{s})$.     

\begin{figure}[h]
    \centering
    \begin{quantikz}[thin lines]
        \lstick{$\ket{0}$} & \gate{\qft_r} & \ctrl{1} & \qw \\
        \lstick{$\ket{\bm{x}}$} & \qw  & \gate{A} & \qw 
    \end{quantikz}
    \caption{The key generation circuit. The gate $\qft_r$ is the quantum fourier transform over $\Z_r$, and the gate $A$ is the multiply-add operation $\ket{j}\ket{\bm{x}} \mapsto \ket{j}\ket{\bm{x} + j\bm{s}}$.}
\end{figure}

\vspace*{\topskip}

$\enc(pk = \rho_{\bm{s}, r, 0}, b \in \{ 0, 1 \})$:  Apply the transformation $U: \ket{j}\ket{\bm{y}} \mapsto (-1)^{bj}\ket{j}\ket{\bm{y}}$ to $\rho_{\bm{s}, r, 0}$ to obtain the state
\begin{align*}
    U \rho_{\bm{s}, r, 0} U^*
    & = U \E_{\bm{x} \leftarrow \Z_q^n} \Big[ \ket{\phi_{\bm{s}, r, 0}(\bm{x})} \bra{\phi_{\bm{s}, r, 0}(\bm{x})} \Big] U^* \\
    & = \E_{\bm{x} \leftarrow \Z_q^n} \Big[ U \ket{\phi_{\bm{s}, r, 0}(\bm{x})} \bra{\phi_{\bm{s}, r, 0}(\bm{x})} U^* \Big] \\
    & = \E_{\bm{x} \leftarrow \Z_q^n} \Big[ \ket{\phi_{\bm{s}, r, b}(\bm{x})} \bra{\phi_{\bm{s}, r, b}(\bm{x})} \Big] \\
    & = \rho_{\bm{s}, r, b}.
\end{align*}
Return $\rho_{\bm{s}, r, b}$.

\begin{figure}[h]
    \centering
    \begin{quantikz}[thin lines]
         & & \lstick{$\ket{j}$} & \ctrl{2} & \qw & \qw \\
         & & \lstick{$\ket{\bm{y}}$} & \qw & \qw & \qw \\
        \lstick{$\ket{0}$} & \gate{X} & \gate{H} & \gate{U_b} & \meter{} & \qw
    \end{quantikz}
    \caption{The encryption circuit. The gate $U_b$ performs the operation $\ket{j}\ket{c} \mapsto \ket{j}\ket{c \oplus (jb \bmod 2)}$.}
\end{figure}

\vspace*{\topskip}

$\dec(sk = \bm{s}, c = \rho_{\bm{s}, r, b})$:  Apply the transformation $S: \ket{j}\ket{\bm{y}} \mapsto \ket{j}\ket{\bm{y} - j\bm{s}}$ to $\rho_{\bm{s}, r, b}$. Discard the second register. Apply $\qft_r$ to the resulting state and measure. If the measurement result is 0 then output 0, otherwise output 1.

\begin{figure}[h]
    \centering
    \begin{quantikz}[thin lines]
        \lstick{$\ket{j}$} & \ctrl{1} & \gate{\qft_r} & \meter{} & \qw \\
        \lstick{$\ket{\bm{y}}$} & \gate{S} & \meter{} & \qw & \qw
    \end{quantikz}
    \caption{The decryption circuit. The gate $\qft_r$ is the quantum fourier transform over $\Z_r$, and the gate $S$ is the multiply-subtract operation $\ket{j}\ket{\bm{x}} \mapsto \ket{j}\ket{\bm{x} - j\bm{s}}$.}
\end{figure}

\begin{lemma}[Correctness]
    For any bit $b \in \{ 0, 1 \}$ and all outputs $(\bm{s}, \rho_{\bm{s}, r, 0})$ of $\gen$, we have
    \[ \Pr [ \dec(\bm{s}, \enc(\rho_{\bm{s}, r, 0}, b)) = b ] = 1. \]
\end{lemma}
\begin{proof}
    Given a ciphertext $\rho_{\bm{s}, r, b}$, the decryption steps are as follows
    \begin{align*}
        \rho_{\bm{s}, r, b}
        & \mapsto \E_{\bm{x} \leftarrow \Z_q^n} \Big[ A \ket{\phi_{\bm{s}, r, b}(\bm{x})} \bra{\phi_{\bm{s}, r, b}(\bm{x})} A^* \Big]  \tag{apply $A$} \\
        & = \E_{\bm{x} \leftarrow \Z_q^n} \bigg[ \frac{1}{r} \sum_{j = 0}^{r - 1} \sum_{k = 0}^{r - 1} (-1)^{b(j - k)}\ket{k}\ket{\bm{x}} \bra{j}\bra{\bm{x}} \bigg] \\
        & = \frac{1}{r} \sum_{j = 0}^{r - 1} \sum_{k = 0}^{r - 1} (-1)^{b(j - k)}\ket{k}\bra{j} \otimes \E_{\bm{x} \leftarrow \Z_q^n} \Big[ \ket{\bm{x}}\bra{\bm{x}} \Big] \\
        & \mapsto \frac{1}{r} \sum_{j = 0}^{r - 1} \sum_{k = 0}^{r - 1} (-1)^{b(j - k)}\ket{k}\bra{j} \tag{discard the second register} \\
        & \mapsto \qft_r \frac{1}{r} \sum_{j = 0}^{r - 1} \sum_{k = 0}^{r - 1} (-1)^{b(j - k)}\ket{k}\bra{j} \qft_r^* \tag{apply quantum Fourier transform}\\
        & = \ket{br/2} \bra{br/2}
    \end{align*}
    If $b = 0$ then the state of the system is $\ket{0} \bra{0}$, otherwise it is $\ket{r / 2} \bra{r / 2}$. 
\end{proof}


\bibliographystyle{plain}
\bibliography{references}
\end{document}



